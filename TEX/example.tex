\documentclass[a5paper,twoside,10pt]{book}
\usepackage[francais,latin]{babel}
\usepackage[OT2,T1]{fontenc}
\usepackage{vmargin}
\setpapersize{A5}
\usepackage{aeguill}
%\usepackage{ae}
%\usepackage{textcomp}
\usepackage{fullpage}
\usepackage{lettrine}
\usepackage{yfonts}
\usepackage{color}
\usepackage{pdfcolmk}
\usepackage{fancyhdr}
\usepackage{parallel}
\usepackage[pdftex]{graphicx}        % Your input file must contain these two lines 
\usepackage{parallel}
\usepackage{lipsum}
\usepackage[pdftex, bookmarks, colorlinks=false, pdftitle={Complies selon le rite romain}, pdfborder={0 0 0}, pdfauthor={Societas laudis}]{hyperref}
\input complies-conf.tex
\begin{document}              

\vspace{0.3cm}
\begin{Parallel}[v]{\colwidth}{\colwidth}
\latin{\vbar\ Deus, in adiutórium meum inténde.\\
\rbar\ Dómine, ad adiuvándum me festína.
Glória Patri, et Fílio, et Spirítui Sancto. 
Sicut erat in princípio, et nunc et semper,
et in sæcula sæculórum. Amen. Allelúia.
}
\vern{\vbar\ Dieu viens à mon aide.\\
\rbar\ Seigneur, vite à mon secours.
Gloire au Père, au Fils et au Saint Esprit. Comme il était au commencement, maintenant et toujours et dans les siècles des siècles. Alléluia.}
\end{Parallel}
\begin{slshape}
\color{rougeliturgique}(Examen de conscience).
\end{slshape}

\begin{Parallel}[v]{\colwidth}{\colwidth}
\latin{
Te lucis ante términum,\\
rerum creátor, póscimus,\\ 
ut sólita cleméntia \\
sis præsul ad custódiam.\\
\vspace*{5pt}
}
\vern{
Traduction de l'hymne.
}
\latin{
Te corda nostra sómnient, \\
te per sopórem séntiant, \\
tuámque semper glóriam \\
vicína luce cóncinant.\\
\vspace*{5pt}
}
\vern{
Traduction de l'hymne.
}
\latin{
Vitam salúbrem tríbue, \\
nostrum calórem réfice \\
tætram noctis calíginem \\
tua collústret cláritas.\\
\vspace*{5pt}
}
\vern{
Traduction de l'hymne.
}
\latin{
Præsta, Pater omnípotens, \\
per Iesum Christum Dóminum, \\
qui tecum in perpétuum \\
regnat cum Sancto Spíritu. Amen.
\vspace*{5pt}
}
\vern{
Traduction de l'hymne.
}
\end{Parallel}

{\color{rougeliturgique}Psalmodie}
{\color{rougeliturgique}Ant. 1} Virgam poténtiæ suæ emíttet Dóminus ex Sion, et regnábit in ætérnum, allelúia.
\begin{Parallel}[v]{\colwidth}{\colwidth}
\latin{{\color{rougeliturgique}Psalmus 109 (110)}}
\vern{{\color{rougeliturgique}Psaume 109 (110)}}
\latin{{\color{rougeliturgique}Messias rex et sacerdos}}
\vern{{\color{rougeliturgique}Le messie, roi et prêtre}}
\latin{\begin{slshape}Oportet illum regnare, donec ponat omnes inimicos sub pedibus eius (1 Cor 15, 25).\end{slshape}}
\vern{\begin{slshape}Il faut qu'Il règne, après avoir mis tous ses ennemis sous ses pieds (1 Cor 15, 25).\end{slshape}}
\latin{Dixit Dóminus Dómino meo: * «Sede a dextris meis, <i>†</i>}
\vern{Le Seigneur a dit à mon Seigneur : * « Assieds-toi à ma droite,}
\latin{donec ponam inimícos tuos * scabéllum pedum tuórum».}
\vern{en attendant que je fasse de tes ennemis l’escabeau de tes pieds ».}
\latin{Virgam poténtiæ tuæ emíttet Dóminus ex Sion: * domináre in médio inimicórum tuórum.}
\vern{Le Seigneur fera sortir de Sion le sceptre de ta puissance : * domine au milieu de tes ennemis.}
\latin{Tecum principátus in die virtútis tuæ, † in splendóribus sanctis, * ex útero ante lucíferum génui te.}
\vern{Avec toi est le principe au jour de ta puissance, †  dans les splendeurs des saints : * c'est de mon sein qu'avant que l'aurore [existât] je t'ai engendré.}
\latin{Iurávit Dóminus et non pænitébit eum: * «Tu es sacérdos in ætérnum secúndum órdinem Melchísedech».}
\vern{Le Seigneur a juré et il ne s'en repentira point : * tu es prêtre pour l'éternité, selon l'ordre de Melchisédech.}
\latin{Dóminus a dextris tuis, * conquassábit in die iræ suæ reges. }
\vern{Le Seigneur est à ta droite : * il a brisé des rois au jour de sa colère.}
\latin{[Iudicábit in natiónibus: cumulántur cadávera, * conquassábit cápita in terra spatiósa.]}
\vern{[Il exercera ses jugements parmi les nations, en accumulant les cadavres, * il écrasera sur la terre les têtes d'un grand nombre.]}
\latin{De torrénte in via bibet, * proptérea exaltábit caput.}
\vern{Il boira du torrent dans le chemin : * c'est pour cela qu'il lèvera la tête.}
\end{Parallel}
Ant. Virgam poténtiæ suæ emíttet Dóminus ex Sion, et regnábit in ætérnum, allelúia.


\end{document}