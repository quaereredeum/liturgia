
\documentclass[twoside,10pt]{book}
\usepackage[francais,latin]{babel}
\usepackage[OT2,T1]{fontenc}
\usepackage{vmargin}
%\setpapersize{A5}
\usepackage{aeguill}
%\usepackage{ae}
%\usepackage{textcomp}
\usepackage{fullpage}
\usepackage{lettrine}
\usepackage{yfonts}
\usepackage{color}
\usepackage{pdfcolmk}
\usepackage{fancyhdr}
\usepackage{parallel}
\usepackage[pdftex]{graphicx}        % Your input file must contain these two lines 
\setlength{\pdfpagewidth}{5.5in}
\setlength{\pdfpageheight}{8.5in}
\usepackage{parallel}
\usepackage{lipsum}

\usepackage[pdftex, bookmarks, colorlinks=false, pdftitle={}, pdfborder={0 0 0}, pdfauthor={}]{hyperref}
\input conf.tex
\begin{document}
\vspace{0.1cm} \begin{Parallel}[v]{\colwidth}{\colwidth}
	\latin{\textnormal{\begin{center}~Hymnus\end{center}}}
	\vern{\textnormal{\begin{center}~Hymne\end{center}}}
	\latin{\textnormal{\begin{center}~O lux beáta cælitum\end{center}}}
	\vern{\textnormal{\begin{center}~Sainte lumère des élus,\end{center}}}
	\latin{\textnormal{\begin{center}~et summa spes mortálium,\end{center}}}
	\vern{\textnormal{\begin{center}~suprême espérance des hommes,\end{center}}}
	\latin{\textnormal{\begin{center}~Iesu, cui doméstica\end{center}}}
	\vern{\textnormal{\begin{center}~Jésus, qui as été accueilli à ta naissance\end{center}}}
	\latin{\textnormal{\begin{center}~arrísit orto cáritas;\end{center}}}
	\vern{\textnormal{\begin{center}~par le sourire d'une famille aimante;\end{center}}}
	\latin{\textnormal{\begin{center}~\end{center}}}
	\vern{\textnormal{\begin{center}~\end{center}}}
	\latin{\textnormal{\begin{center}~María, dives grátia,\end{center}}}
	\vern{\textnormal{\begin{center}~Marie, toute comblée de grâce,\end{center}}}
	\latin{\textnormal{\begin{center}~o sola quæ casto potes\end{center}}}
	\vern{\textnormal{\begin{center}~ô toi la seule qui peux\end{center}}}
	\latin{\textnormal{\begin{center}~fovére Iesum péctore,\end{center}}}
	\vern{\textnormal{\begin{center}~bercer Jésus contre un cœur pur,\end{center}}}
	\latin{\textnormal{\begin{center}~cum lacte donans óscula;\end{center}}}
	\vern{\textnormal{\begin{center}~lui donner ton lait, tes baisers;\end{center}}}
	\latin{\textnormal{\begin{center}~\end{center}}}
	\vern{\textnormal{\begin{center}~\end{center}}}
	\latin{\textnormal{\begin{center}~Tuque ex vetústis pátribus\end{center}}}
	\vern{\textnormal{\begin{center}~Et toi, choisi dans un lignage antique \end{center}}}
	\latin{\textnormal{\begin{center}~delécte custos Vírginis,\end{center}}}
	\vern{\textnormal{\begin{center}~pour être gardien de la Vierge, \end{center}}}
	\latin{\textnormal{\begin{center}~dulci patris quem nómine\end{center}}}
	\vern{\textnormal{\begin{center}~toi que l'enfant divin appelle \end{center}}}
	\latin{\textnormal{\begin{center}~divína Proles ínvocat:\end{center}}}
	\vern{\textnormal{\begin{center}~de ce nom si tendre de père\end{center}}}
	\latin{\textnormal{\begin{center}~\end{center}}}
	\vern{\textnormal{\begin{center}~\end{center}}}
	\latin{\textnormal{\begin{center}~De stirpe Iesse nóbili\end{center}}}
	\vern{\textnormal{\begin{center}~Descendants du noble Jessé,\end{center}}}
	\latin{\textnormal{\begin{center}~nati in salútem géntium,\end{center}}}
	\vern{\textnormal{\begin{center}~vous êtes nés pour le salut des peuples: \end{center}}}
	\latin{\textnormal{\begin{center}~audíte nos, qui súpplices\end{center}}}
	\vern{\textnormal{\begin{center}~écoutez les prières instantes \end{center}}}
	\latin{\textnormal{\begin{center}~ex corde vota fúndimus.\end{center}}}
	\vern{\textnormal{\begin{center}~que nos coeurs font monter vers vous.\end{center}}}
	\latin{\textnormal{\begin{center}~\end{center}}}
	\vern{\textnormal{\begin{center}~\end{center}}}
	\latin{\textnormal{\begin{center}~Qua vestra sedes flóruit\end{center}}}
	\vern{\textnormal{\begin{center}~Qu'il nous soit donné d'imiter, \end{center}}}
	\latin{\textnormal{\begin{center}~virtútis omnis grátia,\end{center}}}
	\vern{\textnormal{\begin{center}~dans la vie de notre maison, \end{center}}}
	\latin{\textnormal{\begin{center}~hanc detur in domésticis\end{center}}}
	\vern{\textnormal{\begin{center}~toutes les vertus dont la grâce \end{center}}}
	\latin{\textnormal{\begin{center}~reférre posse móribus.\end{center}}}
	\vern{\textnormal{\begin{center}~a fleuri votre demeure.\end{center}}}
	\latin{\textnormal{\begin{center}~\end{center}}}
	\vern{\textnormal{\begin{center}~\end{center}}}
	\latin{\textnormal{\begin{center}~Iesu, tuis oboediens\end{center}}}
	\vern{\textnormal{\begin{center}~A toi, Jésus, qui as voulu te faire \end{center}}}
	\latin{\textnormal{\begin{center}~qui factus es paréntibus,\end{center}}}
	\vern{\textnormal{\begin{center}~obéissant à tes parents, \end{center}}}
	\latin{\textnormal{\begin{center}~cum Patre summo ac Spíritu\end{center}}}
	\vern{\textnormal{\begin{center}~à toi, la gloire pour toujours, \end{center}}}
	\latin{\textnormal{\begin{center}~semper tibi sit glória. Amen.\end{center}}}
	\vern{\textnormal{\begin{center}~avec Dieu le Père et l'Esprit.\end{center}}}\end{Parallel}
\begin{Parallel}[v]{\colwidth}{\colwidth}
\latin{ Ant. A porta ínferi érue, Dómine, ánimam meam.
}
\vern{ Ant. Des portes de l’enfer, éloigne mon âme, Seigneur.}
\end{Parallel}
\begin{Parallel}[v]{\colwidth}{\colwidth}
\latin{\textsc{\begin{center}Psalmus 126 (127) \end{center}}}
\vern{\textsc{\begin{center}Psaume 126 (127) \end{center}}}
\latin{\textit{\textbf{\begin{center}Vanus labor sine Domino \end{center}}}}
\vern{\textit{\textbf{\begin{center}Sans le Seigneur, le travail est vain\end{center}}}}
\latin{\textit{\begin{center}Dei ædificatio estis (1 Cor 3, 9).\end{center}}}
\vern{\textit{\begin{center}En Dieu, vous êtes édifiés (1 Cor 3, 9).\end{center}}}
\latin{ Nisi Dóminus ædificáverit domum, * in vanum labórant, qui ædíficant eam.}
\vern{ Si le Seigneur ne bâtit pas la maison, en vain travaillent ceux qui la bâtissent.}
\latin{ Nisi Dóminus custodíerit civitátem, * frustra vígilat, qui custódit eam.}
\vern{ Si le Seigneur ne garde pas la cité, en vain la sentinelle veille à ses portes.}
\latin{ Vanum est vobis ante lucem súrgere †  et sero quiéscere, qui manducátis panem labóris, * quia dabit diléctis suis somnum.}
\vern{ C'est en vain que vous vous levez avant le jour, et que vous retardez votre repos, mangeant le pain de la douleur : il en donne autant à son bien-aimé pendant son sommeil.}
\latin{ Ecce heréditas Dómini fílii, * merces fructus ventris.}
\vern{ Voici, c'est un héritage du Seigneur, que les enfants, une récompense, que les fruits d'un sein fécond.}
\latin{ Sicut sagíttæ in manu poténtis, * ita fílii iuventútis.}
\vern{ Comme les flèches dans la main d'un guerrier, ainsi sont les fils de la jeunesse.}
\latin{ Beátus vir, qui implévit pháretram suam ex ipsis: * non confudétur, cum loquétur inimícis suis in porta.}
\vern{ Heureux l'homme qui en a rempli son carquois. Ils ne rougiront pas quand ils répondront aux ennemis, à la porte de la ville.}\end{Parallel}
\begin{Parallel}[v]{\colwidth}{\colwidth}
\latin{ Ant. A porta ínferi érue, Dómine, ánimam meam.
}
\vern{ Ant. Des portes de l’enfer, éloigne mon âme, Seigneur.}
\end{Parallel}
\end{document}
